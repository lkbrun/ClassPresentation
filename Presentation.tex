\documentclass[12pt,norsk]{beamer}

\usepackage[norsk]{babel}
\usepackage[utf8]{inputenc}
\usepackage{graphicx}

\usetheme{Luebeck}
\usecolortheme{dolphin}

\graphicspath{ {images/} }




%----------------------------------------------------------------------------------------
%	TITLE PAGE
%----------------------------------------------------------------------------------------

\title[Avoiding The Top 10 Software Security Design Flaws]{Avoiding The Top 10 Software Security Design Flaws}
% \subtitle{Chapter 6.1 - 6.2}

\author{Lasse K. Brun \& Haavard Rustad Olsen \\* (Group 5)} % Your name
\institute[UiB]
{
\medskip
\textit{lkbrun@gmail.com \& haavard.olsen@live.com} \\* % Your email address
\textit{INF226 - Software Security, Fall 2014 }
}
\date{13. November 2014} % Date, can be changed to a custom date



\begin{document}

 

\begin{frame}
\titlepage % Print the title page as the first slide
\end{frame}

%----------------------------------------------------------------------------------------
%	PRESENTATION SLIDES
%----------------------------------------------------------------------------------------

\begin{frame}
	\begin{itemize}
		\item Test 1
		\item Test 2
	\end{itemize}		

\end{frame}

\begin{frame}

	\frametitle{1. Earn or Give, but never assume, Trust}
	
	\begin{itemize}
		\item Think about where to implement security functionality. 
		\item Consider all types of clients.
		\item Consider client input compromised until proven otherwise.
		\item Safer in the long run to assume a system not trustworthy.
		\item Design client side to cope with potential compromise.
		\item Validate user input.
		\item Consider where code will be executed, where data will go and where data entering your system comes from. 
	\end{itemize}

\end{frame}

\begin{frame}

	\frametitle{2. Use an authentication mechanism that cannot be bypassed or tampered with}
	\begin{itemize}
	\item One goal of secure design: prevent unauthenticated access
	\item Prevent changing of identity without re-authenticate 
	\item Authenticating requires one or more factors	
	
		\begin{itemize}
			\item Something you know (Password)
			\item Something you are (Biometric signature)
			\item Something you have (Smartphone)
		\end{itemize}
		
	\item Authenticate machines as well as humans	
	
	\end{itemize}
	
	
	
\end{frame}

\begin{frame}

	\frametitle{2. Use an authentication mechanism that cannot be bypassed or tampered with}
	\begin{itemize}
	\item Don't use forgeable session tokens
	\item Use time-tested mechanisms such as Kerberos\cite{kerberos}
	\item Specify time limit for the session if user is inactive
	\item Handle passwords properly!
	\item It's preferable to use on component responsible for authentication
	\end{itemize}
	
\end{frame}

\begin{frame}

	\frametitle{3. Authorize after you authenticate}
	

\end{frame}

\begin{frame}

	\frametitle{4. Strictly separate data and control instructions, and never process control instructions received from untrused sources}
	\begin{itemize}
	\item Co-mingling data and control instructions in a single entity can lead to injection vulnerabilities
	\item General problem in all layers of computer systems
	\item At Lower levels:
	
	\begin{itemize}
		\item Memory-corruption vulnerabilities
		\item May cause attacker controlled modification of control flow.
		\item May cause direct execution of data as machine or bytecode instructions
	\end{itemize}
	
	\end{itemize}

\end{frame}
\begin{frame}
	\frametitle{4. Strictly separate data and control instructions, and never process control instructions received from untrused sources}
	\begin{itemize}
	\item At Higher levels:
	\begin{itemize}
		\item Problems occur in context of runtime interpretation of domain-specific or general-purpose programming languages
		\item If sw assembles string in a parseable language by combining untrusted data with control instructions.
	\end{itemize}
	
	\item You should consider control-flow integrity and separation of control and data as important design goals!
	\item What to consider while designing/using APIs! - Class discussion! 
	\end{itemize}

\end{frame}

\begin{frame}
	\frametitle{4. Strictly separate data and control instructions, and never process control instructions received from untrused sources}
	\begin{itemize}
	\item Designing systems which relies on transforming data into code:
	\begin{itemize}
		\item Take extra care to validate the data as strictly as possible
		\item Areas this is concerning is the \textit{\textbf{eval}} function, reflection or query languages.
	\end{itemize}
	
	\end{itemize}

\end{frame}



\begin{frame}

	\frametitle{5. Define an approach that ensures all data are explicitly validated}
	

\end{frame}

\begin{frame}

	\frametitle{6. Use cryptography correctly}
	

\end{frame}

\begin{frame}

	\frametitle{7. Identify sensitive data and how they should be handled}
	

\end{frame}

\begin{frame}

	\frametitle{8. Always consider the users}
	

\end{frame}

\begin{frame}

	\frametitle{9. Understand how integrating external components changes your attack surface}
	

\end{frame}

\begin{frame}

	\frametitle{10. Be flexible when considering future changes to objects and actors}
	\begin{itemize}
		\item Test \cite{sample}
	\end{itemize}
	
	

\end{frame}

\begin{frame}

	\frametitle{References}
		
	\bibliographystyle{plain}
	\bibliography{biblio}	

\end{frame}




%----------------------------------------------------------------------------------------

\end{document} 