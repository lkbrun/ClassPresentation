\documentclass[t]{beamer}

\mode<presentation> {

% The Beamer class comes with a number of default slide themes
% which change the colors and layouts of slides. Below this is a list
% of all the themes, uncomment each in turn to see what they look like.

\usetheme{default}
%\usetheme{AnnArbor}
%\usetheme{Antibes}
%\usetheme{Bergen}
%\usetheme{Berkeley}
%\usetheme{Berlin}
%\usetheme{Boadilla}
%\usetheme{CambridgeUS}
%\usetheme{Copenhagen}
%\usetheme{Darmstadt}
%\usetheme{Dresden}
%\usetheme{Frankfurt}
%\usetheme{Goettingen}
%\usetheme{Hannover}
%\usetheme{Ilmenau}
%\usetheme{JuanLesPins}
%\usetheme{Luebeck}
%\usetheme{Madrid}
%\usetheme{Malmoe}
%\usetheme{Marburg}
%\usetheme{Montpellier}
%\usetheme{PaloAlto}
%usetheme{Pittsburgh}
%\usetheme{Rochester}
%\usetheme{Singapore}
%\usetheme{Szeged}
%\usetheme{Warsaw}


%\usecolortheme{albatross}
%\usecolortheme{beaver}
%\usecolortheme{beetle}
%\usecolortheme{crane}
%\usecolortheme{dolphin}
%\usecolortheme{dove}
%\usecolortheme{fly}
%\usecolortheme{lily}
%\usecolortheme{orchid}
%\usecolortheme{rose}
%\usecolortheme{seagull}
%\usecolortheme{seahorse}
%\usecolortheme{whale}
%\usecolortheme{wolverine}

%\setbeamertemplate{footline} % To remove the footer line in all slides uncomment this line
%\setbeamertemplate{footline}[page number] % To replace the footer line in all slides with a simple slide count uncomment this line

%\setbeamertemplate{navigation symbols}{} % To remove the navigation symbols from the bottom of all slides uncomment this line
}

\usepackage{graphicx} % Allows including images
\usepackage{booktabs} % Allows the use of \toprule, \midrule and \bottomrule in tables
\usepackage{fontspec}
\usepackage{listings}
\usepackage{xcolor}

\graphicspath{ {images/} }

\lstset{language=SQL}



%----------------------------------------------------------------------------------------
%	TITLE PAGE
%----------------------------------------------------------------------------------------

\title[Avoiding The Top 10 Software Security Design Flaws]{Avoiding The Top 10 Software Security Design Flaws}
% \subtitle{Chapter 6.1 - 6.2}

\author{Lasse K. Brun \& Håvard Rustad Olsen \\* (Group 5)} % Your name
\institute[UiB]
{
\medskip
\textit{lkbrun@gmail.com \& haavard.olsen@live.com} \\* % Your email address
\textit{INF226 - Software Security, Fall 2014 }
}
\date{13. November 2014} % Date, can be changed to a custom date



\begin{document}

 

\begin{frame}
\titlepage % Print the title page as the first slide
\end{frame}

%----------------------------------------------------------------------------------------
%	PRESENTATION SLIDES
%----------------------------------------------------------------------------------------

\begin{frame}
	\begin{itemize}
		\item Test 1
		\item Test 2
	\end{itemize}		

\end{frame}

\begin{frame}

	\frametitle{1. Earn or Give, but never assume, Trust}
	

\end{frame}

\begin{frame}

	\frametitle{2. Use an authentication mechanism that cannot be bypassed or tampered with}
	\begin{itemize}
	\item One goal of secure design: prevent unauthenticated access
	\item Prevent changing of identity without re-authenticate 
	\item Authenticating requires one or more factors	
	
		\begin{itemize}
			\item Something you know (Password)
			\item Something you are (Biometric signature)
			\item Something you have (Smartphone)
		\end{itemize}
		
	\item Authenticate machines as well as humans	
	
	\end{itemize}
	
	
	
\end{frame}

\begin{frame}

	\frametitle{2. Use an authentication mechanism that cannot be bypassed or tampered with}
	\begin{itemize}
	\item Don't use forgeable session tokens
	\item Use time-tested mechanisms such as Kerberos %(http://web.mit.edu/kerberos/)
	\item Specify time limit for the session if user is inactive
	\item Handle passwords properly!
	\item It's preferable to use on component responsible for authentication
	\end{itemize}
	
\end{frame}

\begin{frame}

	\frametitle{3. Authorize after you authenticate}
	

\end{frame}

\begin{frame}

	\frametitle{4. Strictly separate data and control instructions, and never process control instructions received from untrused sources}
	

\end{frame}

\begin{frame}

	\frametitle{5. Define an approach that ensures all data are explicitly validated}
	

\end{frame}

\begin{frame}

	\frametitle{6. Use cryptography correctly}
	

\end{frame}

\begin{frame}

	\frametitle{7. Identify sensitive data and how they should be handled}
	

\end{frame}

\begin{frame}

	\frametitle{8. Always consider the users}
	

\end{frame}

\begin{frame}

	\frametitle{9. Understand how integrating external components changes your attack surface}
	

\end{frame}

\begin{frame}

	\frametitle{10. Be flexible when considering future changes to objects and actors}
	

\end{frame}



%----------------------------------------------------------------------------------------

\end{document} 